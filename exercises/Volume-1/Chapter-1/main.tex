\chapter{Basic Concepts (Chapter 1)}

\section{Algorithms}

\exercise{1}{10}{The text showed how to interchange the values of vairables $m$ and $n$, using the replacement notation, by setting $t \leftarrow m, m \leftarrow n, n \leftarrow t$. Show how the values of $four$ variables $(a,b,c,d)$ can be rearranged to $(b, c, d, a)$ by a sequence of replacements. In other words, the new value of $a$ is to be the original value of $b$, etc. Try to use the minimum number of replacements.}
\label{ex:section1.1-1}
$\begin{array}{lll} t & \leftarrow & a \\a & \leftarrow & b \\ b & \leftarrow & c \\ c & \leftarrow & d \\ d & \leftarrow & t \end{array}$


\exercise{2}{15}{Prove that $m$ is always greater than $n$ at the beginning of step E1, exept possibliy the first time this step occurs.}
\label{ex:section1.1-2}
After the first iteration, we have $r \leftarrow m \mod n, m \leftarrow n, n \leftarrow r$. Since $m \mod n$ is always smaller than $n$, we have $r < m \mod n$.
After the reassignment in Step E3, we now know that $m > n$

\exercise{3}{20}{Change Algorithm E (for the sake of efficiency) so that all trivial replacement operations such as ``$m \leftarrow n$'' are avoided. Write this new algorithm in the style of Algorithm E, and call it Algorithm F}
\label{ex:section1.1-3}

\begin{algorithm}[H]
    \caption{Algorithm F}
    \label{alg:section1.1-3}
    \begin{algorithmic}[1]
        \Require $m,n \in \mathbb{N}$
        \State Divide $m$ by $n$ and let $m$ be the remainder (We will have $0 \leq r < n$)
        \State If $m = 0$ the algorithm terminates and $n$ is the answer.
        \State Divide $n$ by $m$ and let $n$ be the remainder (We will have $0 \leq r < m$)
        \State If $n = 0$ the algorithm terminates and $m$ is the answer.
        \State Go to Step 1

    \end{algorithmic}
\end{algorithm}

\exercise{4}{16}{What is the greatest common divisor of $2166$ and $6099$?}
\label{ex:section1.1-4}
As of algorithm E, we can calculate it:
\begin{itemize}
    \item $m = 2166, n = 6099$, $r = 2166 \mod 6099 = 2166$
    \item $m = 6099, n = 2166$, $r = 6099 \mod 2166 = 1773$
    \item $m = 2166, n = 1773$, $r = 2166 \mod 1773 = 399$
    \item $m = 1773, n = 399$, $r = 1773 \mod 399 = 171$
    \item $m = 399, n = 171$, $r = 399 \mod 171 = 57$
    \item $m = 171, n = 57$, $r = 171 \mod 57 = 0$
\end{itemize}
So the greatest common divisor is $57$

\exercise{5}{12}{Show that the ``Procedure for Reading This Set of Books'' that appears after the preface actually fails to be a genuine algorithm on at least three of our five counts! Also mention some differences in format between it and Algorithm E.}
\label{ex:section1.1-5}
\todo{solution (Page 9)}

\exercise{6}{20}{What is $T_5$, the average number of times step E1 is performed when $n=5$?}
\label{ex:section1.1-6}
We know that after the first iteration, $m$ will be $5$ and $n < 5$ (since $m \mod n$ is always smaller than $n$). So we can calculate the average number of times step E1 is performed by calculating the average number of iterations when $n \in \{0,1,2,3,4\}$.
\begin{align*}
    T_5 & = 1 + \left(\frac{1}{5} \cdot T_1 + \frac{1}{5} \cdot T_2 + \frac{1}{5} \cdot T_3 + \frac{1}{5} \cdot T_4 + \frac{1}{5} \cdot T_0 \right) \\
        & = 1 + \left(\frac{1}{5} \cdot 1 +   \frac{1}{5} \cdot 2 +   \frac{1}{5} \cdot 3 +   \frac{1}{5} \cdot 2   + \frac{1}{5} \cdot 0\right)    \\
        & = 2.6                                                                                                                                     \\
\end{align*}

\exercise{7}{HM21}{Let $U_m$ be the average number of times step E1 is performed in algorithm E if $m$ is known and $n$ is allowed to range over all positive integers. Show that $U_m$ is well defined. Is $U_m$ in any way related to $T_m$?}
\label{ex:section1.1-7}
We have three cases. $m > n$,  $n < m$ and $m = n$ \\

First, we examine the trivial case $m = n$: In this case, $r = 0$ and the algorithm terminates. So $U_m = 1$. \\

Now we examine the equally trivial case $m < n$: In this case, $r = m$ and the algorithm switches $m$ and $n$ in the first iteration. So $U_m = 1 + T_m$.

Finally, we examine the case $m > n$: This case has only finite possibilities (exactly $m$ possibilities). This means that $m > n$ is ignored in the average.

\exercise{8}{M25}{Give an ``effective'' formal algorithm for computing the greatest common divisor of positive integers $m$ and $n$, by specifying $\theta_j, \phi_j, a_j, b_j$ as in Eqs. (3). Let the input be represented by the string $a^mb^n$, that is $m$ $a$'s followed by $n$ $b$'s. Try to make your solution as simple as possible. [Hint: Use Algorithm E, but instead of division in step E1, set $r \leftarrow |m - n|, n \leftarrow \min(m,n)$.]}
\label{ex:section1.1-8}
\todo{solution (Page 9)}

\exercise{9}{M30}{Suppose that $C_1 = (Q_1, I_1, \Omega_1, f_1)$ and $C_2 = (Q_2, I_2, \Omega_2, f_2)$ are computational methods. For example $C_1$ might stand for Algorithm E as in Eqs. (2), except that $m$ and $n$ are restricted in magnitude, and $C_2$ might stand for a computer program that $m$ and $n$ are restricted in magnitude, and $C_2$ might stand for a computer program implementation of Algorithm E. (Thus $Q_2$ might be the set of all states in a machine, i.e., all possible configurations of its memory and registers: $f_2$ might be the definition of single machine actions; and $I_2$ might be the set of initial states, each including the program that determines the greatest common divisor as well as the particular values of $m$ and $n$.) \\
    Formulate a set-theoretic definition for the concept ``$C_2$ is a representation of $C_1$.''  or ``$C_2$ simualtes $C_1$.'' This is to mean intuitively that any computation sequence of $C_1$ is mimicked by $C_2$, except that $C_2$ might take more steps in which to do the computation and it might retain more information in its states (We thereby obtain a rigorous interpretation of the statement, ``Program $X$ is a representation of Algorithm $Y$.'')}
\label{ex:section1.1-9}
\todo{solution (Page 9)}


\section{Mathematical Induction}

\exercise{1}{05}{Explain how to modify the idea of proof by mathematical induction, in case we want to prove some statement $P(n)$ for all \textit{nonnegative} integers --- that is, for $n = 0,1,2,\ldots$ instead of for $n = 1,2,3,\ldots$ .}
\label{ex:section1.2-1}
Prove that $P(0)$ is true. \\
Prove that if $P(n)$ is true, then $P(n+1)$ is true. \\

\exercise{2}{15}{
There must be something wrong with the following proof. What is it? \\
``\textit{Theorem.} Let $a$ be any positive number. For all positive integers $n$ we have $a^{n-1} = 1$. \\
\textit{Proof.} If $n=1, a^{n-1} = a^{1-1} = a^0 = 1.$ And by induction assuming the theorem is true for $1,2,\ldots, n$ we have
$$a^{(n+1)-1} = a^n = \frac{a^{n-1} \times a^{n-1}}{b^{n-1}} = \frac{1 \times 1 }{1} = 1, \hspace{1em}\text{where } b=a^{(n-2)/(n-1)};$$%
so the theorem is true for $n+1$ as well.''}
\label{ex:section1.2-2}
If we expand the definition of $b$ we get $b = a^{(n-2)/(n-1)} = a^{(n-2) \cdot (n-1)^{-1}}$
Than we can use this in $a^n = \frac{a^{n-1} \times a^{n-1}}{b^{n-1}} = a^n = \frac{a^{n-1} \times a^{n-1}}{(a^{(n-2) \times (n-1)^{-1}})^{n-1}} = \frac{a^{n-1} \times a^{n-1}}{a^{n-2}}$.
For the case $n = 1$ we have a division by zero. So this is no valid proof.

\exercise{3}{18}{The following proof by induction seems correct, but for some reason the equation for $n = 6$ gives $\frac12 + \frac 16 + \frac1{12} + \frac1{20} + \frac1{30} = \frac56$ on the left-hand side, and $\frac32 - \frac16 = \frac43$ on the right-hand side. Can you find a mistake? \\ \\
    ``\textit{Theorem.}
    $$\frac{1}{1 \cdot 2} + \frac{1}{2 \cdot 3} + \dots + \frac{1}{(n-1)\cdot n} = \frac{3}{2} - \frac{1}{n}$$

    \textit{Proof.} We use induction on $n$. For $n = 1$, clearly $\frac{3}{2} - \frac{1}{n} = \frac{1}{1 \cdot 2}$; and, assuming that the theorem is true for $n$,
    \begin{align*}
        \frac{1}{1 \cdot 2} + \dots + \frac{1}{(n-1)\cdot n} + \frac{1}{n \cdot (n+1)} &                                                                                                      \\
        = \frac{3}{2} - \frac{1}{n} + \frac{1}{n \cdot (n+1)}                          & = \frac{3}{2}- \frac{1}{n} + \left(\frac{1}{n} - \frac{1}{n+1}\right) = \frac{3}{2}- \frac{1}{n+1}''
    \end{align*}
    %
}
\label{ex:section1.2-3}

The problem is, that the induction base case is not true. For $n=1$ we have $\frac{1}{(n-1)\cdot n} = \frac{1}{0 \cdot 1}$, which is undefined.

\exercise{4}{20}{Prove that, in addition to Eq. (3), Fibonacci numbers satisfy $F_n \geq \phi^{n-2}$ for all positive integers $n$}
\label{ex:section1.2-4}
We know that $\phi = \frac{1+\sqrt{5}}{2}$ and $\phi + 1 = \phi^2$. \\

\textbf{Base case:} $n = 1$ \\
$F_1 = 1 \geq \phi^{1-2} = \phi^{-1} = \frac{1}{\phi} = \frac{2}{1+\sqrt{5}} \approx 0.61$ \\
$F_2 = 1 \geq \phi^{2-2} = \phi^0 = 1$ \\
$F_3 = 2 \geq \phi^{3-2} = \phi = \frac{1+\sqrt{5}}{2} \approx 1.61$ \\

\textbf{Induction step:} Assume that $F_n \geq \phi^{n-2}$ for an arbitrary, but fixed $n \in \N$. \\
$F_{n+1} \stackrel{!}{=} \phi^{n-1}$ \\
$F_{n+1} = F_n + F_{n-1} \geq \phi^{n-2} + \phi^{(n-1)-2} = \phi^{n-2} + \phi^{n-3} = (\phi + 1)\cdot\phi^{n-3} = \phi^2\cdot \phi^{n-3} = \phi^{n-1} \checkmark$
$\square$

\exercise{5}{21}{A prime number is an integer $> 1$ that has no positive integer divisors other than $1$ and itself. Using this definition and mathematical induction, prove that every integer $> 1$ can be written as a product of one or more prime numbers. (A prime number is considered to be the ``product'' of one single prime, namely itself.)}
\label{ex:section1.2-5}

\textbf{Base case:} $n = 2$ \\
$2$ is prime. Thus $2$ can be written as $2$, which is a product of primes.

\textbf{Induction step:} Assume that for all $i \in \{1, ..., n-1\}$ $i$ can be written as product of primes.
To show: $n$ can be written as product of primes.

Case: $n$ is prime. Thus $n$ can be written as product of primes, namely $n$. \\

Case: $n$ is not a prime. So there exists a positive integer $q < n$ such $q|n$.
We can deduct that $n = p \cdot q$ with $q<n,p<n$. We know this are primes or can be written as product of primes ($p_1, ..., p_m$ and $q_1, ..., q_{m'}$). So $n$ can be writte as product $p_1 \cdot ... \cdot p_m \cdot q_1 \cdot ... \cdot q_{m'}$, which are primes.

$\square$

\exercise{6}{20}{Prove that if Eqs. (6) hold just before step E4, they hold afterwards also.}

If Eqs. 6 holds before step E4, than $a'm+b'n = c$ and $am+bn = d$ \\
The reordering is as following: $c \leftarrow d, d \leftarrow r, t \leftarrow a', a' \leftarrow a, a \leftarrow t -qa, t \leftarrow  b', b' \leftarrow b, b \leftarrow t-qb$
After the reordering we want to show, that $a'm+b'n = c$ and $am+bn = d$ is still true.
If we plugin in this formula the original values we get:
$am + bn = d$ (which is still true) and $(a'-qa)m + (b'-qb)n = r$ \\
We know that $r = c \mod d$ and $q = \lfloor \frac cd \rfloor$ (or said in other words: $c = qd + r$ for $0 \leq r < d$ and $q > 0$)

So in original values we get: $(a'-qa)m + (b'-qb)n = c \mod d$
\begin{align*}
    (a'-qa)m + (b'-qb)n         & = c \mod d                  \\
    a'm-qam +b'n -qbn           & = c \mod d & | a'm +b'n = c \\
    -qam-qbn                    & = 0 \mod d                  \\
    -q(\underbrace{am+bn}_{=d}) & = 0 \mod d                  \\
\end{align*}
So both equations are still true.

\exercise{7}{23}{Formulate and prove by induction a rule for the sums $1^2, 2^2-1^2, 3^2-2^2+1^2, 4^2-3^2+2^2-1^2$, etc.}
Let the sum be known as $f(n)$ with $n$ is the largest base. \\
So $f(n) := n^2 - (n-1)^2 + ... - (-1)^n 1^2$
\begin{align*}
    f(n) & = n^2 - (n-1)^2 + ... - (-1)^n 1^2       \\
         & = \sum_{i = 0}^{n} (-1)^{i}\cdot (n-i)^2 \\
    g(n) & = \sum_{i=0}^{n} i = \frac{n(n-1)}{2}    \\
\end{align*}
We will show, that $f(n) = g(n)$ for all $n \geq 1$: \\

\textbf{Induction base}: \\
$f(1) = 1^2 = 1 = g(1)$ and $f(2) = 2^2-1^2 = 3 = 2+1 = g(n)$

\textbf{Induction step}: \\
Let $n$ be an arbitrary but fixed positive integer and for all $i < n$ it is true that $f(i) = g(i)$
It is to be shown that: $f(n) = g(n)$ \\
$f(n) = n^2 - f(n-1)$ and $g(n) = n + g(n-1) = \frac{n(n-1)}{2}$ \\
Easily we can see, that: $f(n) = n^2 - \sum_{i=0}^{n-1}(-1)^i ((n-1)-i)^2 = n^2 -\frac{(n-1)(n-2)}{2} = \frac{n^2+3n+2}{2}$

\todo{solution (Page 19)}

\exercise{8}{25}{(a) Prove the following theorem of Nicomachus (A.D. c. 100) by induction: $1^3 = 1, 2^3 = 3+5, 3^3 = 7+9+11, 4^3 = 13+15+17+19,$ etc. \\
    (b) Use this result to prove the remarkable formula $1^3+2^3+\ldots + n^3 = (1+2+\ldots + n)^2$}
\label{ex:section1-2:8}
(a): \\
If $n$ is even: $n^3 = n^2\cdot n = \underbrace{n^2 + \ldots + n^2}_{n-times} = (n^2-\frac{n}{2}) + (n^2-\frac{n}{2}+2) + \ldots + (n^2-1) + (n^2+1) + \ldots + (n^2-\frac{n}{2}-2) + (n^2 + \frac{n}{2})$, so the Theorem applies for even numbers. \\ \\
If $n$ is odd: $n^3 = n^2\cdot n = \underbrace{n^2 + \ldots + n^2}_{n-times} = (n^2-\lceil\frac{n}{2}\rceil) + (n^2-\lceil\frac{n}{2}\rceil+2) + \ldots + (n^2) + \ldots + (n^2-\lceil\frac{n}{2}\rceil-2) + (n^2 + \lceil\frac{n}{2}\rceil)$, so the Theorem applies for odd numbers. \\ \\

(b): \\
To show is: $\sum_{i = 1}^n i^3 = \left(\sum_{i=1}^{n} i \right)^2$ \\
We already know: $\sum_{i=1}^{n} i = \frac{n(n-1)}{2}$, so $\left(\sum_{i=1}^{n} i \right)^2 = \left( \frac{n(n-1)}{2}\right)^2$

\textbf{Induction base}: $n = 1$ \\
$\sum_{i = 1}^1 i^3 = 1$ and $\left(\sum_{i = 1}^1 i\right)^2 = 1$.
It is obvious, that $1=1$ is a true statement.


\textbf{Induction step}: Let $n$ be an arbitrary but fixed positive integer and for all $j \leq n$ it is true that $\sum_{i = 1}^j i^3 = \left(\sum_{i=1}^{j} i \right)^2$

\begin{align*}
    \sum_{i = 1}^{n+1} i^3                 & \stackrel{!}{=} \left(\sum_{i=1}^{n+1} i \right)^2                                                             \\
    \sum_{i = 1}^{n} i^3 + (n+1)^3         & = \left(\sum_{i=1}^{n} i  + (n+1)\right)^2                                                                     \\
    \sum_{i = 1}^{n} i^3 + (n^3+3n^2+3n+1) & = \left(\sum_{i=1}^{n} i\right)^2 + 2\cdot \left(\sum_{i=1}^{n} i\right) \cdot(n+1)  + (n+1)^2 & | - \text{IV} \\
    (n^3+3n^2+3n+1)                        & =  2\cdot \left(\sum_{i=1}^{n} i\right) \cdot(n+1)  + (n+1)^2                                                  \\
    (n^3+3n^2+3n+1)                        & =  2\cdot \left(\frac{n(n+1)}{2}\right) \cdot(n+1)  + (n+1)^2                                                  \\
    (n^3+3n^2+3n+1)                        & =  (n^3+2n^2+n)  + (n^2+2n+1)                                                                                  \\
    (n^3+3n^2+3n+1)                        & =  n^3+3n^2+3n+1 \checkmark                                                                                    \\
\end{align*}

\exercise{9}{20}{Prove by induction that if $0<a<1$,then $(1-a)^n \geq 1-na$ for all integers $n$}
\label{ex:section1-2:9}

\textbf{Induction Base}: $n = 0$ \\
$(1-a)^0 = 1$, $1-0 \cdot 0 = 1$, $1=1 \checkmark$ \\

\textbf{Induction Step ($n \rightarrow n+1$):} Let $n$ be an arbitrary but fixed positive integer and for all $0 \leq j \leq n$ it is true that $(1-a)^j \geq 1-ja$ \\ \\

We know: $(1-a)^n \geq 1-na$ \\
So after the induction Step we get: $(1-a)^{n+1} \geq 1-(n+1)a$ \\
\begin{align*}
    (1-a)^{n+1}           & \geq 1-(n+1)a                 \\
    (1-a)^{n} \cdot (1-a) & \geq 1-na-a                   \\
    (1-a)^{n} - (1-a)     & \geq 1-na-a   & | - \text{IV} \\
    a+1                   & \geq -a                       \\
\end{align*}
The last Statement is tue ($a$ is positive).

\textbf{Induction Step ($n \rightarrow n-1$):} Let $n$ be an arbitrary but fixed positive integer and for all $n \leq j \leq 0$ it is true that $(1-a)^j \geq 1-ja$ \\ \\

\begin{align*}
    a+1                   & \geq -a       & | + \text{IV} \\
    (1-a)^{n} - (1-a)     & \geq 1-na-a                   \\
    (1-a)^{n} \cdot (1-a) & \geq 1-na-a                   \\
    (1-a)^{n+1}           & \geq 1-(n+1)a                 \\
\end{align*}


So $(1-a)^n \geq 1-na$ is true for all integers $n$ if $a \in (0,1)$.

\exercise{10}{M22}{Prove by induction that if $n \geq 10$, then $2^n > n^3$}
\textbf{Induction Base ($n=10$):} \\
$2^10 = 1024$, $10^3 = 1000$, $1024 > 1000$ \\ \\

\textbf{Induction Step ($n \rightarrow n+1$):} Let $n$ be an arbitrary but fixed positive integer and for all $n \geq j \geq 10$ it is true that $2^j > j^3$ \\ \\
\begin{align*}
    2^{n+1}     & > (n+1)^3                   \\
    2^n \cdot 2 & > n^3+ 3n^2 + 3n +1 & |- IV \\
    2^n         & > 3n^2 +3n + 1
\end{align*}
We already know, that: $2^n > n^3$ and $n^3 > 3n^2 +3n+1$, so it is true.

\exercise{11}{M30}{Find and prove a simple formula for the sum
    $$\frac{1^3}{1^4+4} - \frac{3^3}{3^4+4} + \frac{5^3}{5^4+4} - \ldots + \frac{(-1)^n(2n+1)^3}{(2n+1)^4+4}$$}
\todo{Solution (Page 19)}

\exercise{12}{M25}{Show how Algorithm E can be generalized as stated in the text so it will accept input values of the form $u+v\sqrt{2}$, where $u$ and $v$ are integers, and the computations can still be done in an elementary way (that is, without using the infinite decimal expansion of $\sqrt{2}$). Prove that the computation woll not terminate, however, if $m = 1$ and $n = \sqrt{2}$}

\todo{Solution (Page 19)}

\exercise{13}{M23}{Extend Algorithm E by adding a new Variable $T$ and adding the operation ``$T \leftarrow T+1$'' at the beginning of eacht step. (Thus, $T$ is like a clock, counting the number of steps executed.) Assume that $T$ is initially zero, so that assertion $A1$ in Fig. 4 becomes ``$m > 0, n > 0, T = 0.$'' The additional condition ``$T=1$'' should similarly be appended to $A2$. Show how to append additional conditions to the assertions in such a way that any one of $A1, A2, \ldots, A6$ implies $T\leq 3n$, and such that the inductive proof can still be carried out. (Hence the computations must terminate in at most $3n$ steps.)  }

\todo{Solution (Page 20)}

\exercise{14}{50}{(R.W. Floyd.) Prepare a computer program that accepts, as input, programs in some programming language together with optional assertions, and that attempts to fill in the remaining assertions necessary to make a proof that the computer program is valid. (For example, strive to get a program that is able to prove the validity of Algorithm E, given only assertions $A1, A4,$ and $A6$. See tge oaoers by R.W. Floyd and J. C. King in the IFIP Congress proceedings, 1971, for further discussion.) }

\exercise{16}{HM28}{See Page 20}

