\section{Algorithms}

\exercise{1}{10}{The text showed how to interchange the values of vairables $m$ and $n$, using the replacement notation, by setting $t \leftarrow m, m \leftarrow n, n \leftarrow t$. Show how the values of $four$ variables $(a,b,c,d)$ can be rearranged to $(b, c, d, a)$ by a sequence of replacements. In other words, the new value of $a$ is to be the original value of $b$, etc. Try to use the minimum number of replacements.}
\label{ex:section1.1-1}
$\begin{array}{lll} t & \leftarrow & a \\a & \leftarrow & b \\ b & \leftarrow & c \\ c & \leftarrow & d \\ d & \leftarrow & t \end{array}$


\exercise{2}{15}{Prove that $m$ is always greater than $n$ at the beginning of step E1, exept possibliy the first time this step occurs.}
\label{ex:section1.1-2}
After the first iteration, we have $r \leftarrow m \mod n, m \leftarrow n, n \leftarrow r$. Since $m \mod n$ is always smaller than $n$, we have $r < m \mod n$.
After the reassignment in Step E3, we now know that $m > n$

\exercise{3}{20}{Change Algorithm E (for the sake of efficiency) so that all trivial replacement operations such as ``$m \leftarrow n$'' are avoided. Write this new algorithm in the style of Algorithm E, and call it Algorithm F}
\label{ex:section1.1-3}

\begin{algorithm}[H]
    \caption{Algorithm F}
    \label{alg:section1.1-3}
    \begin{algorithmic}[1]
        \Require $m,n \in \mathbb{N}$
        \State Divide $m$ by $n$ and let $m$ be the remainder (We will have $0 \leq r < n$)
        \State If $m = 0$ the algorithm terminates and $n$ is the answer.
        \State Divide $n$ by $m$ and let $n$ be the remainder (We will have $0 \leq r < m$)
        \State If $n = 0$ the algorithm terminates and $m$ is the answer.
        \State Go to Step 1

    \end{algorithmic}
\end{algorithm}

\exercise{4}{16}{What is the greatest common divisor of $2166$ and $6099$?}
\label{ex:section1.1-4}
As of algorithm E, we can calculate it:
\begin{itemize}
    \item $m = 2166, n = 6099$, $r = 2166 \mod 6099 = 2166$
    \item $m = 6099, n = 2166$, $r = 6099 \mod 2166 = 1773$
    \item $m = 2166, n = 1773$, $r = 2166 \mod 1773 = 399$
    \item $m = 1773, n = 399$, $r = 1773 \mod 399 = 171$
    \item $m = 399, n = 171$, $r = 399 \mod 171 = 57$
    \item $m = 171, n = 57$, $r = 171 \mod 57 = 0$
\end{itemize}
So the greatest common divisor is $57$

\exercise{5}{12}{Show that the ``Procedure for Reading This Set of Books'' that appears after the preface actually fails to be a genuine algorithm on at least three of our five counts! Also mention some differences in format between it and Algorithm E.}
\label{ex:section1.1-5}
\tasktodo

\exercise{6}{20}{What is $T_5$, the average number of times step E1 is performed when $n=5$?}
\label{ex:section1.1-6}
We know that after the first iteration, $m$ will be $5$ and $n < 5$ (since $m \mod n$ is always smaller than $n$). So we can calculate the average number of times step E1 is performed by calculating the average number of iterations when $n \in \{0,1,2,3,4\}$.
\begin{align*}
    T_5 & = 1 + \left(\frac{1}{5} \cdot T_1 + \frac{1}{5} \cdot T_2 + \frac{1}{5} \cdot T_3 + \frac{1}{5} \cdot T_4 + \frac{1}{5} \cdot T_0 \right) \\
        & = 1 + \left(\frac{1}{5} \cdot 1 +   \frac{1}{5} \cdot 2 +   \frac{1}{5} \cdot 3 +   \frac{1}{5} \cdot 2   + \frac{1}{5} \cdot 0\right)    \\
        & = 2.6                                                                                                                                     \\
\end{align*}

\exercise{7}{HM21}{Let $U_m$ be the average number of times step E1 is performed in algorithm E if $m$ is known and $n$ is allowed to range over all positive integers. Show that $U_m$ is well defined. Is $U_m$ in any way related to $T_m$?}
\label{ex:section1.1-7}
We have three cases. $m > n$,  $n < m$ and $m = n$ \\

First, we examine the trivial case $m = n$: In this case, $r = 0$ and the algorithm terminates. So $U_m = 1$. \\

Now we examine the equally trivial case $m < n$: In this case, $r = m$ and the algorithm switches $m$ and $n$ in the first iteration. So $U_m = 1 + T_m$.

Finally, we examine the case $m > n$: This case has only finite possibilities (exactly $m$ possibilities). This means that $m > n$ is ignored in the average.

\exercise{8}{M25}{Give an ``effective'' formal algorithm for computing the greatest common divisor of positive integers $m$ and $n$, by specifying $\theta_j, \phi_j, a_j, b_j$ as in Eqs. (3). Let the input be represented by the string $a^mb^n$, that is $m$ $a$'s followed by $n$ $b$'s. Try to make your solution as simple as possible. [Hint: Use Algorithm E, but instead of division in step E1, set $r \leftarrow |m - n|, n \leftarrow \min(m,n)$.]}
\label{ex:section1.1-8}
\tasktodo

\exercise{9}{M30}{Suppose that $C_1 = (Q_1, I_1, \Omega_1, f_1)$ and $C_2 = (Q_2, I_2, \Omega_2, f_2)$ are computational methods. For example $C_1$ might stand for Algorithm E as in Eqs. (2), except that $m$ and $n$ are restricted in magnitude, and $C_2$ might stand for a computer program that $m$ and $n$ are restricted in magnitude, and $C_2$ might stand for a computer program implementation of Algorithm E. (Thus $Q_2$ might be the set of all states in a machine, i.e., all possible configurations of its memory and registers: $f_2$ might be the definition of single machine actions; and $I_2$ might be the set of initial states, each including the program that determines the greatest common divisor as well as the particular values of $m$ and $n$.) \\
    Formulate a set-theoretic definition for the concept ``$C_2$ is a representation of $C_1$.''  or ``$C_2$ simualtes $C_1$.'' This is to mean intuitively that any computation sequence of $C_1$ is mimicked by $C_2$, except that $C_2$ might take more steps in which to do the computation and it might retain more information in its states (We thereby obtain a rigorous interpretation of the statement, ``Program $X$ is a representation of Algorithm $Y$.'')}
\label{ex:section1.1-9}
\tasktodo
